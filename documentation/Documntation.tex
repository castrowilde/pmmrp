\documentclass[12pt,leqn]{article}

\title{Обработка на изображения чрез реакционно-дифузни модели}
\author{Пламен Никифоров, Християн Марков, Стефан Велинов}
\date{\today}

\usepackage[T2A]{fontenc}
\usepackage{amsmath,cmap}
\usepackage[cp1251]{inputenc}
\usepackage[english,bulgarian]{babel}

\begin{document}
\maketitle

\tableofcontents

\section{Въведение}

Целта на този проект е реализиране на обработка на изображения

We propose a new concept of image processing by use
of self-organization mechanisms appeared in a discrete
non-linear system. Edge detection, image segmentation,
noise reduction and contrast enhancement can be achieved
by use of a discrete reaction-diffusion model (Fitz-Hugh
& Nagumo model) under the condition of Turing
instability. Compared with the conventional method, the
proposed one indicates a higher performance in
processing for noisy image. Especially, by adding a
suitable level of noise, the model works as contrast
enhancement. It is confirmed that the non-linear effect of
stochastic resonance (SR) brings a good performance in
image processing. 

\section{Дифузия}
Нека първо добием представа какво представлява дифузията. Най-просто казано дифузията е
процес, при който някакво вещество или енергия се разпространява от зони с по-висока концентрация,
към такива с по-ниска. За да стане по-ясно как точно работи дифузията нека разгледаме
следните примери.\\

\textbf{Пример 1}: Представете си, че имаме метален прът и започнем да нагряваме
единия му край. Нагрятият край ще има по-висока концентрация на топлина и посредством дифузията, топлината
ще започне да се пренася по дължината на пръта, към края с по-ниска концентрация на топлина, до достигане на равновесно положение в което ще имаме една и съща температура по целия прът.\\\\

\textbf{Пример 2}: Нека имаме аквариум пълен с вода. Пускайки количество мастило в аквариума, ние увеличаваме концентрацията на мастило там, където сме го пуснали спрямо останалата част на аквариума. Мастилото ще започне да се разнася към тези части с по-ниска концентрация, до достигане на положение, в което концентрацията на мастило във водата е равномерно разпределена.\\

Вече имайки тази интуитивна представа за това какво представлява дифузията, нека разгледаме процеса по-строго от научна гледна точка и да го опишем математически. За да постигнем тази цел, ще трябва да разгледаме закона за запазване и закона на Фик.

\section{Реакция}

\end{document}
